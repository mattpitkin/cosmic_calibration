\IfFileExists{revtex4-1.cls}{\documentclass[prd, twocolumn, lengthcheck,
superscriptaddress, showpacs, letterpaper, nofootinbib]{revtex4-1}}{
\IfFileExists{revtex4.cls}{\documentclass[prd, twocolumn, lengthcheck,
superscriptaddress, showpacs, letterpaper, nofootinbib]{revtex4}}{}
}

\usepackage{latexsym}
\usepackage{graphicx}

\usepackage{color}

\usepackage{amsmath}
\usepackage{amssymb}
\usepackage{hyperref}
\usepackage{bm}
\usepackage{acronym}
\usepackage{url}
\usepackage[normalem]{ulem}   % temporal one in draft.
\usepackage{xspace}

\usepackage{amsopn} % for using DeclareMathOperator

% variable shortcuts 
\newcommand{\Hub}{H_{0}}
\newcommand{\DL}{D_{\mathrm{L}}}
\newcommand{\lam}{\bm{\lambda}}

\newcommand{\curlH}{\mathcal{H}}
\newcommand{\gws}{\tilde{h}}
\newcommand{\scf}{\ensuremath{\mathcal{C}}}

\newcommand{\btheta}{\mathbf{\theta}}

%% ----- input git-version tag
\input{tag.tex}

\newcommand{\dcc}{LIGO-PXXXXXXXX}
\newcommand{\cm}[1]{\textcolor{red}{CM: #1}}
\newcommand{\MP}[1]{\textcolor{blue}{MP: #1}}
\newcommand{\lw}[1]{\textcolor{green}{LW: #1}}

\DeclareMathOperator{\erf}{erf}

% define macro for PASP from aas_macros.sty
\newcommand\pasp{PASP}

\begin{document}

\title{Astrophysical calibration of gravitational-wave detectors}

\author{L.~Wright}\affiliation{SUPA, School of Physics and Astronomy, University of
  Glasgow, Glasgow G12 8QQ, United Kingdom}
\author{M.~Pitkin}\affiliation{SUPA, School of Physics and Astronomy, University of
  Glasgow, Glasgow G12 8QQ, United Kingdom}
\author{C.~Messenger}\affiliation{SUPA, School of Physics and Astronomy, University of
  Glasgow, Glasgow G12 8QQ, United Kingdom}

\email{matthew.pitkin@glasgow.ac.uk}
\email{christopher.messenger@glasgow.ac.uk}

\date{\today}

\begin{abstract}
  We present an investigation into the potential of assessing the validity of the 
calibration of gravitational wave detector outputs through the use of standard
  sirens. Such signals, as measured via gravitational wave
  observations provide an estimated luminosity distance which is
  subject to uncertainties in the calibration of the data.  If a host
  galaxy is identified for a given source then its redshift can be
  obtained and combined with current knowledge of the cosmological
  parameters.  This will yield the true luminosity distance and allow
  the direct comparison with the estimated value.  Discrepancies can
  then be used to correct the original calibration.~\cm{Update at end}
\end{abstract}

\pacs{04.80.Nn, 95.55.Ym}
\preprint{LIGO-P15XXXXX}

\maketitle

% acronym definitions
\acrodef{GW}[GW]{gravitational wave}
\acrodef{NS}[NS]{neutron star}
\acrodef{BNS}[BNS]{binary neutron star}
\acrodef{MWEG}[MWEG]{Milky Way equivalent galaxies}
\acrodef{BBH}[BBH]{binary black hole}
\acrodef{BNS}[BNS]{binary neutron star}
\acrodef{NSBH}[NSBH]{neutron star-black hole}
\acrodef{SNR}[SNR]{signal-to-noise ratio}
\acrodef{GRB}[GRB]{gamma-ray burst}
\acrodef{sGRB}[sGRB]{short gamma-ray burst}
\acrodef{CBC}[CBC]{compact binary coalescence}
\acrodef{EM}[EM]{electro-magnetic}
\acrodef{aLIGO}[aLIGO]{Advanced LIGO}
\acrodef{AdV}[AdV]{Advanced Virgo}
\acrodef{PSD}[PSD]{power spectral density}
\acrodef{MCMC}[MCMC]{Markov chain Monte Carlo}
\acrodef{ET}[ET]{Einstein Telescope}
\acrodef{XRT}[XRT]{x-ray telescope}

%%%%%%%%%%%%%%%%%%%%%%%%%%%%%%%%%%%%%%%%%%%%%%%%%%
%%%%%%%%%%%%%%%%%%%%%%%%%%%%%%%%%%%%%%%%%%%%%%%%%%
\section{Introduction\label{sec:intro}}

% introduce GWs and BNS systems
It is expected that the advanced generation of interferometric \ac{GW} detectors will detect waves 
emitted from $O(10s)$ of \acp{CBC}. One such class of these cataclysmic events, the inspiral and
merger of \ac{BNS} systems will be detected out to a maximum range of $\approx 450$ Mpc. Assuming 
the current best estimates for the cosmological parameters, this is equivalent to a redshift 
$z\approx 0.1$. As noted by \cite{1986Natur.323..310S} \ac{CBC} systems can be used as cosmological 
distance markers, otherwise known as ``standard sirens'' (analogous to the \ac{EM} standard 
candles). Standard sirens allow ...

% BNS systems are GRBs
It is considered likely that the merger of \ac{BNS} systems, in addition to emitting detectable 
\acp{GW}, is also the mechanism for producing \acp{sGRB}. In this scenario these \ac{EM} events are 
produced through the ... and result in tightly beamed emission parallel to the orbital angular 
momentum vector of the \ac{BNS} system. Additional evidence for the coincidence of \acp{sGRB} with
\ac{GW} events are the estimated astrophysical rates of both phenomena. Observations of \acp{sGRB} 
give a rate of XXX Mpc$^{-3}$yr$^{-1}$ which can be compared to estimates of \ac{BNS} merger rates 
of XXX-XXX Mpc$^{-3}$yr$^{-1}$. This latter range is obtained from population synthesis models 
together with knowledge of the X known \ac{BNS} systems in our galaxy. 
  
% talk about the calibration idea
If a single \ac{GW} event is observed in coincidence with a \ac{sGRB} it may be possible to 
identify the host galaxy of the \ac{BNS} source.  With this information a spectroscopic redshift 
can be very accurately obtained. With knowledge of the redshift, using current best estimates for 
the Hubble constant and other cosmological parameters, the true luminosity distance can be
estimated to $\sim 1\%$ accuracy.  The \ac{GW} measurement acts as a standard siren also giving us 
a direct measurement of the luminosity distance to the source. The accuracy of such a measurement 
depends on a number of factors including the accuracy with which the \ac{GW} detector has been 
calibrated.  Hence comparison with the distance estimate from the \ac{sGRB} we can re-calibrate (or 
validate) the existing experimentally obtained calibration. 

% What do we do in this paper
In this paper we investigate the feasibility of this approach for single coincident 
\ac{GW}--\ac{sGRB} events and establish the validation power of such a calibration technique.  In 
Sec.~\ref{sec:calibration} we briefly summarize the existing experimental technique for \ac{GW} 
detector calibration and its expected accuracy.  We then review the concept of \ac{GW} standard 
sirens and their proposed coincident \ac{EM} signatures, the \acp{sGRB} in
Secs.~\ref{sec:sirens} and~\ref{sec:GRB}. In Sec.~\ref{sec:cosmo} we show the likely accuracy of 
\ac{sGRB} distances obtained from their host galaxy redshift and in Sec.~\ref{sec:discussion} we 
conclude.    

%%%%%%%%%%%%%%%%%%%%%%%%%%%%%%%%%%%%%%%%%%%%%%%%%%
%%%%%%%%%%%%%%%%%%%%%%%%%%%%%%%%%%%%%%%%%%%%%%%%%%
\section{Gravitational wave detector calibration\label{sec:calibration}}

The technique used throughout \ac{GW} research in calibrating the detectors is carried out through 
a complicated system of physical manipulations of the Fabret-Perot Michelson interferometer; where 
an elaborate feedback system is used to sustain a defined measurement in arm length difference 
between the moving mirrors. A comprehensive description of the calibration procedure (in particular 
for the LIGO detectors during their fifth science run) can be found in \cite{2010NIMPA.624..223A}, 
and a brief description is giving in \cite{Vitale:2012} and references therein. For \ac{aLIGO} 
and \ac{AdV}, at the frequency range used in this project, the error in the amplitude calibration 
using the `hardware injection method' is roughly $10\%$ \cite{Vitale:2012}. This is a benchmark set 
for the error estimation using the proposed new method in this project.

% Describe our model for the calibration
The \ac{GW} strain is measured through the differential arm length, $\Delta L$, changes of the 
interferometer via $h(f,t) = \Delta L(f,t) / L$, where $L$ is the full arm length. Calibration is 
required to relate the actual measured interferometer error signal output $e(f)$ to $\Delta L$. 
This relation is known as the length response function, $R(f)$, defined such that
\begin{equation}
\Delta L(f,t) = R(f) e(f, t),
\end{equation}
where where we assume $R$ varies only slowly in time (in comparison to transient signal 
time-scales). Calculation of $R$ requires the measurement of various functions within a control 
feedback loop (see \cite{2010NIMPA.624..223A}, which are subject to uncertainties. In this 
study we will assume an estimate of $R$ is available (although in theory we could take on the role 
of estimating $R$ itself), but that it differs from the truth through some unknown scale factor, 
$\scf$, so that
\begin{equation}\label{eq:scalefactor}
\scf h(f) = h_{\rm m}(f) = \frac{R(f,t) e(f)}{L},
\end{equation}
where $h(f)$ is the true strain and $h_{\rm m}(f)$ is the measured strain. With this definition it
means that if $\scf > 1$ then a signal would appear to have a larger amplitude (e.g.\ be 
closer) than in reality\footnote{However, the \ac{SNR} would be the correct value as the signal and 
the noise will both contain the scale factor}, whereas if $\scf < 1$ it would appear to have a 
lower amplitude than reality. In this analysis we will simplify the situation by assuming that 
$\scf$ is a constant with respect to frequency, but in future studies that assumption could be 
dropped and $\scf$ could take some functional form, or piece-wise fit, with respect to $f$ (e.g.\ 
in a similar way to the method of \cite{2013PhRvD..88h4044L} used for fitting differences in 
\ac{PSD} estimates).

%%%%%%%%%%%%%%%%%%%%%%%%%%%%%%%%%%%%%%%%%%%%%%%%%%
%%%%%%%%%%%%%%%%%%%%%%%%%%%%%%%%%%%%%%%%%%%%%%%%%%
\section{Binary neutron star standard sirens\label{sec:sirens}}

% Standard siren basics
The idea of \ac{GW} standard sirens is directly analogous to the concept of
standard candles in \ac{EM} astronomy and was first proposed
in~\cite{1986Natur.323..310S}. Unlike the primary \ac{EM} standard candle
event, Type 1a Supernovae, the measured luminosity of a \ac{CBC} in \acp{GW} is
not only a function of distance. It also depends upon the chirp mass (a
function of the component masses) and the binary orientation with respect to
the detector network. However, measurement of the phase evolution of such an
event allows accurate determination of the chirp mass~\footnote{In reality it
is the redshifted chirp mass that is measured.}. Timing information from the
different signal arrival times at each detector in the network allows
reasonably accurate sky position determination.  Finally, amplitude variation
between differently oriented interferometers allows some (weaker) level of
determination of the binary system orientation.  A great help to this final
stage for this project is the assumption that the source has an \ac{sGRB}
counterpart.  It follows that we therefore have a well constrained prior belief
on the inclination of the source orbit with respect to the line-of-sight based
on the expected level of beaming in the \ac{sGRB} emission.

% The classic use of sirens
Standard sirens are clearly a very powerful tool for \ac{GW} cosmology since
they give us a direct measure of the absolute luminosty distance to sources.
This is distinct from Type 1a Supernovae standard candles that only provide
relative luminosty distance measures and require calibration via other methods
as part of the cosmological distance ladder. At first sight the missing
component for standard sirens is a complementary measurement of the source
redshift which would allow each event to inform the distance-redshift
relationship and hence obtain cosmological parameter estimates.  However, it
was not thought possible (until recently) that redshift measurements could be
obtained from \ac{GW} observations. In~\cite{DelPozzo:2011uf} (building on the
original idea proposed in~\cite{1986Natur.323..310S} the idea of identifying
host galaxies within the sky position uncertainty region was developed. Each
galaxy would provide a redshift measure and would contribute to a disparate mix
of possible cosmological parameters. However, by combining multiple (${\sim}20$)
detections, the true set of cosmological parameters emerges with an expected
${\sim}5\%$ uncertainty on the Hubble constant. 

% continued use of sirens
The use of \ac{GW} events with \ac{sGRB} counterparts was investigated
in~\cite{2010CQGra..27u5006S} and~\cite{2011PhRvD..83b3005Z} with respect to
the 3rd generation \ac{GW} interferometer, the \ac{ET}. It was
assumed in both studies that 1000 coincident events would be observed and
cosmological parameter estimation ability was compared with current \ac{EM}
results with focus on dark energy parameters. Other methods for cosmological
inference (see~\cite{1996PhRvD..53.2878F},~\cite{2012PhRvD..85b3535T},
and~\cite{2012PhRvD..86b3502T}) have been proposed for \ac{BNS} systems that do
not rely on any redshift measurements and instead use the distribution of
measured \acp{SNR} combined with some assumptions regarding the \ac{NS} mass
and spatial distributions. Finally we mention that for \ac{BNS} systems it has
been shown in~\cite{Messenger:2011ux} and~\cite{2013arXiv1312.1862M} that features
of the tidal and post-merger hyper-massive \ac{NS} stages of the \ac{GW}
waveform can allow the measurement of the source redshift if the \ac{NS}
equation-of-state is sufficiently well known. This development, and seperately, the method
based on \ac{SNR} distributions will allow \ac{GW} cosmological inference
without the need for complementary \ac{EM} observations.

% put this in context with calibration
To put the existing work on \ac{GW} cosmology in context with the study
described in this manuscript we are essentially turning the standard
cosmological problem upside-down. We assume that the existing \ac{EM}
cosmological parameter estimation (specifically for the Hubble constant) is
accurately determined and therefore a known quantity. We then assume that the
unknown in our analysis is the absolute amplitude calibration of the
interferometers in the global \ac{GW} detector network. The issue of
calibration has largely been ignored in \ac{GW} cosmological inference studies
and in Sec.~\ref{sec:discussion} we discuss some of the implications of
combining these issues in future analyses.     

%%%%%%%%%%%%%%%%%%%%%%%%%%%%%%%%%%%%%%%%%%%%%%%%%%
%%%%%%%%%%%%%%%%%%%%%%%%%%%%%%%%%%%%%%%%%%%%%%%%%%
\section{GRB counterparts\label{sec:GRB}}

% what are sGRBs 
It is believed that \acp{sGRB} (those \acp{GRB} with duration ${<}2$ sec) are
emitted during the merger of
\acp{BNS}~\cite{1989Natur.340..126E},~\cite{1992ApJ...395L..83N}. The emmission
from these events is highly beamed along the binary rotation axis and hence
only potentially observable for a fraction of \ac{BNS} mergers. If the event
exhibits an optical afterglow then the host galaxy can be identified from which
a redshift can be obtained.  The fraction of events with associated redshifts
is ${\sim}1/3$. The range to which the \emph{Swift} \ac{XRT} has detected
\acp{sGRB} is $z{\sim}2$ and the nearest event with associated redshift is at
$z{sim}0.1$~\cite{2015GCN..17278...1C} which is approximately equal to the
horizon range of the advanced \ac{GW} detector network.    


% this is copied text for reference !! 
%Estimates for jet beaming are θj ∼ 5◦ for lGRBs and θj ∼ 5 − 15◦ for sGRBs
%(Burrows et al. 2006, Grupe et al. 2006, Fong et al. 2012). Beaming angles for
%sGRBs are still highly uncertain. The beaming factors fb = 1 − cos θj ≃ θj2/2
%are roughly 1/300 for lGRBs and 1/30 for sGRBs. Based on the observed rate of
%sGRBs by Swift, Coward et al. (2012) estimate a LVC detection rateof∼3−30yr−1
%forθj ≃ 15◦. Chen&Holz(2013)claim3−7yr−1 forGRBGW+EM detections. Kelley et al.
%(2013) estimate the rate of Swift or Fermi observations joint with LVC
%detections to be ∼0.07 yr−1. Siellez, Boe ̈r, & Gendre (2014) consider current
%and future high energy missions and estimate a rate of simultaneous GW+EM
%detections of ∼0.1 − 4 yr−1 in the LVC era. Wanderman & Piran (2015) estimate a
%co-detection rate LVC+Fermi of 0.1 − 1 yr−1 and LVC+Swift of 0.02 − 0.14 yr−1.

% discuss joint-event rates
\cm{Discuss the expected rate of joint detections}\\

% observing a coincident event
The most likely scenario in which a coincident \ac{GW}--\ac{sGRB} event would be identified is 
through the targeted follow-up of an observed \ac{GW} or by post-facto matching of \ac{GW} 
trigger lists with known \ac{sGRB} events.  The likelihood of being able to follow-up \ac{GW} 
events using gamma-ray telescopes with low enough latency to catch a \ac{sGRB} is low.  Compounding 
this issue is the relatively large \ac{GW} sky error-box giving a field-of-view for the \ac{EM} 
observatories to search spanning $\sim 100$'s of square degrees~\cite{grb}.  For the \ac{GW} 
follow-up of \ac{sGRB} scenario the merger time for \ac{BNS} systems will be estimated from the 
\ac{sGRB} to within a few seconds~\cite{grb}.  This makes the follow-up search less computationally
expensive since it is performed over a smaller range of data using potentially fewer numbers of 
waveform templates.  This computational saving enables the use of a more computationally expensive 
multi-detector coherent scheme rather than the cheaper coincidence methods used in the untargeted 
searches. 

% discuss the inclination angle issue
A fortunate consequence of a joint \ac{GW}--\ac{sGRB} observation will be the fact that in order 
for such an event to be observed, the \ac{BNS} system must have had its orbital angular momentum 
vector pointing towards (or away) from the detector.  The actual inclination angle of the system, 
defined as the angle between the orbital angular momentum vector and the line of sight, must be $<$
half of the \ac{sGRB} beaming angle.  This prerequisite property limits us to systems that are 
approximately ``face-on'' and therefore biases us to higher \ac{SNR} signals.  However, as 
discussed earlier, the property of beaming severely impacts the probable rate of such joint 
observations.   


%%%%%%%%%%%%%%%%%%%%%%%%%%%%%%%%%%%%%%%%%%%%%%%%%%
%%%%%%%%%%%%%%%%%%%%%%%%%%%%%%%%%%%%%%%%%%%%%%%%%%
\section{Analysis}

We want to assess how well the calibration scale factor defined in Eqn.~\ref{eq:scalefactor} can be 
estimated from a single observed \ac{GW} associated with a particular \ac{sGRB}. As we {\it a 
priori} have no knowledge of the likely location and distance of such an event we have performed 
simulations of multiple events to see the expected distribution in the accuracy of the calibration 
scale factor recovery.

% Describe the signal model (Taylor F2 etc...)
We use the TaylorF2 waveform approximation (see e.g.\ \cite{2009PhRvD..80h4043B} and references 
therein) with a 3.5 post-Newtonian expansion in phase for modelling our signal (in both simulations 
and signal recovery).  For \ac{BNS} systems we use non-spinning waveforms under the assumption that 
spins will be negligible. For \ac{NSBH} system we use a spinning waveform, but in which only the 
black hole has non-negligible spin and that its rotational angular momentum is aligned with the 
system's orbital angular momentum.

For a non-spinning system the general form of the frequency-domain polarisation amplitudes are
%
\begin{eqnarray}\label{eq:signal}
  \gws_{+}(f) &\propto \frac{1+\cos^{2}\iota}{D_{L}}
\mathcal{M}^{5/6}f^{-7/6}e^{-i\Psi(f, \nu, t_\mathrm{c}, \phi_{\mathrm{c}})} \nonumber \\
  \gws_{\times}(f) &\propto
 \frac{\cos^{2}\iota}{D_{L}}\mathcal{M}^{5/6} f^{-7/6}e^{-i\Psi(f, \nu, t_\mathrm{c}, 
\phi_{\mathrm{c}})}
\end{eqnarray}
%
where the chirp-mass $\mathcal{M}$ is defined as $\mathcal{M}=M\eta^{3/5}$, with 
$\eta=m_{1}m_{2}/M^{2}$ and $M=m_{1}+m_{2}$, $\iota$ is the inclination angle, and $\nu=(\pi 
Mf)^{1/3}$.  The \ac{GW} strain measured at the $k^{\rm{th}}$ detector is then given by
%
\begin{equation}
  \label{eq:gravsig}
   \gws^{k}(f) = \big[ F_{+}^{k}(\alpha, \delta, \psi)\gws_{+}(f) +
F_{\times}^k(\alpha, \delta, \psi)\gws_{\times}(f)\big]
\end{equation}
%
where $F_{+}$ and $F_{\times}$ are the antenna response functions which are dependent upon the 
polarisation angle $\psi$ and the sky position of the source $\alpha$ and $\delta$.

% Define the detectors and the noise epoch we will be using
For this analysis we will consider the advanced generation \ac{aLIGO} and \ac{AdV} detectors 
operating at their design sensitivity, with noise power spectral densities taken from
\cite{2013arXiv1304.0670L}. The network we consider is the two \ac{aLIGO} detectors (H1 and L1) and 
the \ac{AdV} detector (V1). 

% Define the likelihood function we will be using
We consider that the measured data in any detector is defined as
%
\begin{equation}
  \tilde{d}_{k}(f)=\scf_{k}(\tilde{n}_{k}(f)+\gws_{k}(f,\btheta))
\end{equation}
%
where $\tilde{n}(f)$ is the Fourier transform of the true (in our case Gaussian) strain noise drawn 
from a, $\btheta$ is the set of waveform parameters and $\scf_k$ is the calibration scale factor for 
the $k^{\rm th}$ detector.

\subsection{Method}

To assess the ability to estimate the unknown calibration scale factors we want to calculate their 
marginal posterior probability distribution functions (pdf). We use Bayes' theorem for which we 
need to define a likelihood function and prior pdfs on the signal parameters being estimated. In 
this analysis we use the Markov chain Monte Carlo code {\tt emcee} \cite{2013PASP..125..306F} to 
sample the posterior distribution. We use a Gaussian likelihood function given by
\begin{widetext}
\begin{equation}
p(\bm{d} | \btheta, \vec{\scf}, \curlH, I) \propto \exp{\left[4\Delta f \sum_{k=1}^{N_{\rm 
det}}
  \sum_{i=i_{f_{\rm low}}}^{i_{f_{\rm high}}}\frac{\Re{\left\{ \tilde{d}^{*}_{k,i} 
\scf_{k}\gws_{k,i}(\btheta) - \frac{1}{2}\left(\scf_k^2 \gws^*_{k,i}(\btheta)\gws_{k,i}(\btheta) + 
\tilde{d}^*_{k,i}\tilde{d}_{k,i}\right) \right\}}}{S_{k,i}}\right]}
\end{equation}
\end{widetext}
where $\bm{d}$ is an array containing the data for all detectors, $\Delta f$ is the frequency 
resolution, $S_k$ is the $k^{\rm th}$ detector's measured one sided noise \ac{PSD} (and as such 
will contain the effect of the calibration scale factor), and the $i$ indices increment over 
frequency with $i_{f_{\rm low}}$ and $i_{f_{\rm high}}$ corresponding to the lower 
and upper range in frequencies used for our analysis, which were 20\,Hz to 400\,Hz respectively. 
This frequency range was chosen as the vast majority of the \ac{SNR} for the inspiralling signals 
we use can be found within this range, so going to lower or higher frequencies provides very little 
additional information. The source parameters have been subsumed into a vector $\btheta$ and the 
calibration scale factors are within the vector $\vec{\scf}$.

\subsubsection{Prior ranges}\label{sec:priors}

The primary reason why we can use coincident observations with a \ac{sGRB} as a check of detector 
calibration is that the \ac{sGRB} allows us to constrain the priors on various parameters that 
would normally be highly correlated with the signal amplitude. The most important of these is 
that the \ac{sGRB} can provide a very tight constraint on the source distance\footnote{For \ac{BNS} 
and \ac{NSBH} systems they will be observable only in the relatively nearby Universe ($z \sim 
0.1$ and $z\sim 0.2$ for \ac{BNS} and \ac{NSBH} systems respectively), so the relation between the 
measured \ac{sGRB} redshift and distance is simple and our luminosity distance will be negligibly 
effected by the choice of cosmology \MP{Check this?}.}. In this analysis we therefore assume that 
the source distance is known (i.e.\ has a $\delta$-function prior). Another piece of information 
that we can use from having a coincident \ac{sGRB} is that the system is likely to be relatively 
close to face-on (and therefore nearly circularly polarised), which allows us to place prior 
constraints on the system inclination angle. Beam opening angles for \acp{sGRB} are relatively 
poorly constrained as they are based on only a few jet break detections, however 
\cite{2014ApJ...780..118F} give a median opening angle value of $\sim 10^{\circ}$. If we assume 
that the beam opening angle is a half-normal distribution, for which the median is given by 
$\sqrt{2}\erf{}^{-1}(1/2)\sigma \sim 0.67\sigma$, then a median of $\sim 10^{\circ}$ corresponds to 
$\sigma \sim 14.8^{\circ}$. We therefore place a Gaussian prior on the inclination angle with zero 
mean and $\sigma=14.8^{\circ}$.

Our prior on the system component masses is dependent on whether we are considering the source 
being a \ac{BNS} system or a \ac{NSBH}. For the former case the prior we use is a Gaussian 
distribution for both components with means of $1.35\,\textrm{M}_{\odot}$ and standard deviations 
of $0.13\,\textrm{M}_{\odot}$ \MP{(reference)}. In reality, for all the systems we use in our 
study, the chirp mass, which we see from equation~\ref{eq:signal} plays a role in the overall 
signal amplitude, is very well constrained by the system's phase evolution, so our prior could be 
expanded with minimal effect on the results. In the case of \ac{NSBH} systems we use the same 
prior as above for the neutron star, but for the black hole we use a prior based on the 
canonical masss distribution used for the rate results in \cite{2012PhRvD..85h2002A} with a mean of 
$5\,\textrm{M}_{\odot}$ and a standard deviation of $1\,\textrm{M}_{\odot}$. We note that the mass 
range of black holes could be quite different from this, but use this range as an example for this 
type of system. For the \ac{NSBH} systems we also require a prior on the black hole spin (for 
\ac{BNS} systems we have assumed that the spins are small enough that they will be negligible). We 
use a uniform prior on the normalised aligned spin magnitude between $-1$ and $1$.

The final important piece of information that we can make use of from the \ac{sGRB} observation is 
the sky position of the source, which we take to be known precisely.

For the time of coalescence we use a uniform prior of spanning $\pm0.01$\,seconds around the 
recorded time of the signal that would be returned by the detection pipeline. For the reference 
phase a uniform prior between 0 and $\pi$ is used, and for the polarisation angle a uniform prior 
between 0 and $\pi/2$ is used. As the signals we use are generally close to being circularly 
polarised (face-on) the phase and polarisation angle will be degenerate meaning the exact 
combination of them will have very little effect of the result.

Finally, we require a prior on the calibration scale factors for each detector. We make the 
assumption that in general the calibration applied to the detector data will be correct, so want to 
use a prior that is peaked at unity. We also assume that the probability density that the 
calibration scale factor is either larger or smaller than unity by an equivalent factor, e.g.\ 
$s=10$ or $s=0.1$, is the same. For the scale factor for each detector we we therefore use a 
log-normal distribution as our prior
\begin{equation}
 p(\scf|I) = \frac{1}{\scf\sigma\sqrt{2\pi}}\exp{\left( -\frac{(\ln{\scf} - \mu)^2}{2\sigma^2} 
\right)},
\end{equation}
where we chose $\sigma = 1.07$, which in turn means that having the mode at unity gives $\mu = 
1.15$. With these parameters the prior probability density at 0.1 and 10 is one tenth of that at 
1. Note that this prior does not give a symmetric amount of probability about unity, but given our 
simulation criterion described below in Section~\ref{sec:simulations} we find that the likelihood 
generally overwhelms the prior. As $\scf$ is a scale factor we also have the further constraint 
that it must be positive.

\subsection{Simulations}\label{sec:simulations}

To estimate how well we can constrain the calibration scaling for each detector in an advanced 
detector network we have performed simulations with injected \ac{BNS} and \ac{NSBH} signals at a 
range of distances (50\,Mpc to 500\,Mpc with 50\,Mpc increments for the \ac{BNS} signals and 
100\,Mpc to 900\,Mpc with 100\,Mpc increments for the \ac{NSBH} signals). The simulations used the 
two \ac{aLIGO} detectors (H1 and L1) and the \ac{AdV} detector (V1). The simulations were all 
performed in the frequency domain and spanned a frequency range from 20\,Hz to 400\,Hz. We assumed
all detectors were operating at their design sensitivity (as given by Figure~1 of 
\cite{2013arXiv1304.0670L}). As such we added noise to each simulation based on the sensitivity 
curve, but scaled with the associated calibration scale factor.

% Describe the ensemble of simulations used
Separately for the \ac{BNS} and \ac{NSBH} systems we have performed a large number simulations at 
each distance value. When proposing injections for each simulation the source parameters were 
randomly drawn from the prior distributions given in Section~\ref{sec:priors}, with the sky 
position being drawn randomly from a uniform distribution on the sky, the coalescence times held 
fixed within the centre of the prior window, and with the calibration scaling factors for each 
detector drawn from a Gaussian distribution with a mean of one and standard deviation of $0.125$ 
(equivalent to a mean fractional offset of 10\%). However, in accepting a proposed injection as one 
to be analysed 
we introduced a criteria that the signal be ``detectable'', based on that used in 
\cite{2012PhRvD..85h2002A} of it having an \ac{SNR} of $\geq 5.5$ in at least two detectors (as 
these would be signals coincident with a \ac{sGRB} we do not include the often used further 
constraint that the total coherent network \ac{SNR} is greater than 12). This criteria means that 
at larger distances our population of injected sources was not uniformly distributed over the sky, 
and were generally closer to being circularly polarised.

These simulations have allowed us to assess how well on average we would be able to constrain the 
calibration scale for a given \ac{CBC}-\ac{sGRB} coincidence.

\section{Results}\label{sec:results}

A simple assumption that one could make would be that the uncertainties on the calibration scale 
factors (if they are independent of other parameters and have a roughly Gaussian probability 
distribution) should be given by $\sim 1/{\rm SNR}$ for each detector. As we will show below this 
assumption is reasonable, but small correlations do exist between parameters meaning that it does 
not completely hold.

For each of the simulated sources we have used the {\tt emcee} python \ac{MCMC} package 
\cite{2013PASP..125..306F} to perform parameter estimation over the unknown source parameters using 
the priors as discussed in Section~\ref{sec:priors}. In each case when calculating the likelihood 
we used an estimate of the noise \ac{PSD} based on the advanced detector design sensitivities 
(using those given in \cite{2013arXiv1304.0670L}), but simulated as if calculated by averaging 64 
separate noisy \ac{PSD} estimates and scaled with the same factor as applied to the injection and 
noise\footnote{We do not account for there being a potential difference between the estimated 
\ac{PSD} and the actual \ac{PSD} of the analysed section of data as described in e.g.\ 
\cite{2013PhRvD..88h4044L}. This difference would be very highly correlated with the calibration 
scale factor, so in reality our estimate of the scale factor would be a combination of the
calibration offset and the difference in the \ac{PSD}. As such our results on real data would be an 
upper limit on the calibration scale factor.}. This has provided posterior probability distributions 
on the calibration scale factors for each detector. Examples of the marginalised posterior 
probabilities for a \ac{BNS} system and a \ac{NSBH} system observed with the three detector network 
are shown in figures~\ref{fig:bnspost} and \ref{fig:nsbhpost} respectively.

From these posterior distributions we have calculated the minimal 68\% credible region on the 
calibration scale factors for each detector (if these were Gaussian distributions this is 
equivalent to the region either side of the mean bounded by the $1\sigma$ intervals). For all the 
signals at each distance increment we have produced the distribution of the fractional half widths 
(i.e.\ $1\sigma$) of these scale factors' confidence intervals compared to the actual value. These 
are shown as boxplots in figures~\ref{fig:bnsresults} and \ref{fig:nsbhresults} for the \ac{BNS} 
and \ac{NSBH} systems respectively. The boxes show the extent from the lower to upper quartile of 
the values, whilst the whiskers extend from the 5$^{\rm th}$ to 95$^{\rm th}$ percentile. The black 
line within each box gives the median value and the star gives the mean value. Also shown on each 
plot as the dashed magenta line is the percentage of signals drawn from the prior distribution that 
fulfil the detection criterion.

\begin{figure*}
 \begin{center}
  \includegraphics[width=1.0\textwidth]{bns_post_fig.pdf}
 \end{center}
 \caption{\label{fig:bnspost} The marginalised posterior probability distributions for the unknown
 parameters of a \ac{BNS} system, including calibration scaling factors for the three detectors 
(H1, L1 and V1). The injected signal was at a distance of 250\,Mpc, had \ac{SNR} of 8.7, 10.9 and 
4.1 and percentage uncertainties in the calibration scale factors of 14\%, 11\% and 27\% for each 
of the detectors respectively.}
\end{figure*}

\begin{figure*}
 \begin{center}
  \includegraphics[width=1.0\textwidth]{nsbh_post_fig.pdf}
 \end{center}
 \caption{\label{fig:nsbhpost} The marginalised posterior probability distributions for the unknown
 parameters of a \ac{NSBH} system, including calibration scaling factors for the three detectors 
(H1, L1 and V1). The injected signal was at a distance of 450\,Mpc, had \ac{SNR} of 7.7, 10.9 and 
4.2 and percentage uncertainties in the calibration scale factors of 17\%, 14\% and 34\% for each of 
the detectors respectively.}
\end{figure*}

\begin{figure*}
 \begin{center}
  \includegraphics[width=1.0\textwidth]{scale_factor.pdf}
 \end{center}
 \caption{\label{fig:bnsresults} Distributions of the percentage accuracy at which the calibration 
scale factors can be determined for a three detector network if assuming \ac{BNS} systems (provided 
a coincident \ac{GRB} is observed and can yield a distance estimate). The boxplots span the lower 
to upper quartile range of the distributions, with the median value shown as a horizontal line 
within the box and the mean shown as a star. The dashed magenta line shows the percentage of source 
drawn from the prior distribution that would be detectable at each distance value.}
\end{figure*}

\begin{figure*}
 \begin{center}
  \includegraphics[width=1.0\textwidth]{scale_factor_nsbh.pdf}
 \end{center}
 \caption{\label{fig:nsbhresults} Distributions of the percentage accuracy at which the calibration 
scale factors can be determined for a three detector network if assuming \ac{NSBH} systems 
(provided a coincident \ac{GRB} is observed and can yield a distance estimate). The plot contents 
are the same as in figure~\ref{fig:bnsresults}.}
\end{figure*}

In figure~\ref{fig:bnsresults} we see that on average the scale factor can be recovered to 
equivalent precision in both \ac{aLIGO} detectors, as would be expected, with uncertainties 
generally within the 10\% range for sources at 100\,Mpc. This is comparable to previous estimates 
of the calibration error in the initial LIGO detectors. An interesting feature is that for 
distances $\gtrsim 250$\,Mpc the upper extent of the uncertainty for H1 and L1 hits a maximum at 
$\sim 25\%$, whilst the width of boxes narrows. This is due to our ``detectability'' criteria, 
in that for all distances we will only be seeing those sources with a high enough \ac{SNR} that we 
would consider them detectable. There will therefore be fewer sources with \ac{SNR} higher than 
this criteria at large distances, giving us a narrower range, and we automatically exclude those 
with \ac{SNR} that are too small thus truncating our uncertainty distribution at the upper end. 
However, this does show that on average for sources that are detectable out to 450\,Mpc we would be 
able to constrain the calibration scale factor uncertainty for the \ac{aLIGO} detectors to $\sim 
20\%$. As the detection criteria will generally come from the \ac{SNR} being high enough within the 
two \ac{aLIGO} instruments this means that the SNR in \ac{AdV} can be small and thus the ability to 
constrain the scale factor for it becomes poor (although it still provides information that the 
calibration is not grossly inaccurate). We also see that true uncertainties achievable for 
recovering the calibration scale factors are indeed very similar to the simple assumption that they 
would be given by $\sim 1/{\rm SNR}$, i.e.\ at the upper end of the distribution the \ac{SNR} in 
the two \ac{aLIGO} detectors will be $\sim 5.5$, which would be expected to produce uncertainties 
of $\lesssim 20\%$. Our results are probably slightly poorer than the \ac{SNR} expectation due to 
their still be correlations with $\iota$ even within the constrained prior range, and slight 
correlations between the scale factors for H1 and L1 as seen in figure~\ref{fig:bnspost}.

In figure~\ref{fig:nsbhresults} we see very similar results for the \ac{NSBH} systems although the 
higher \ac{SNR} of the signals means that we can provide $\sim 20\mbox{--}25\%$ uncertainties on 
the calibration scale factors for H1 and L1 out to greater distances. The distributions generally 
appear slightly broader than those for the \ac{BNS} systems. This may be entirely explicable 
through two factors. The first factor is from the fact that including the spin parameter leads to 
strong correlations between the chirp mass, mass ratio and spin. These strong correlations make 
convergence of the \ac{MCMC} take longer and means we have fewer independent samples to estimate 
the posterior distributions with. This leads to a larger statistical fluctuation on the results. 
The second factor is that there appears to be a modest effect due to the population of sources that 
are detectable. For distances at which the \ac{BNS} and \ac{NSBH} systems would give the same 
percentage of observable sources there is a slight increase in the mean and standard distribution 
of \acp{SNR} of the \ac{NSBH} systems over the \ac{BNS} systems.

As noted previously the mass distribution for the black holes could be quite different from the one 
we used, with masses extending to $\gtrsim 10$\,M$_{\odot}$. We expect these would produce similar 
results to those we see in figure~\ref{fig:nsbhresults}, but again extending to further distances. 
Including the merger and ring-down phase, which we currently ignore, may also help increase 
\ac{SNR}.

%%%%%%%%%%%%%%%%%%%%%%%%%%%%%%%%%%%%%%%%%%%%%%%%%%
%%%%%%%%%%%%%%%%%%%%%%%%%%%%%%%%%%%%%%%%%%%%%%%%%%
\section{Distance estimates from \acp{sGRB}\label{sec:cosmo}}

~\cm{Describe the cosmological analysis used to obtain the distance from the
GRB}

%%%%%%%%%%%%%%%%%%%%%%%%%%%%%%%%%%%%%%%%%%%%%%%%%%
%%%%%%%%%%%%%%%%%%%%%%%%%%%%%%%%%%%%%%%%%%%%%%%%%%
\section{Multiple events without counterpart\label{sec:multiple}}

\cm{A discussion of the issues regarding multiple events
without EM counterparts.}

%%%%%%%%%%%%%%%%%%%%%%%%%%%%%%%%%%%%%%%%%%%%%%%%%%
%%%%%%%%%%%%%%%%%%%%%%%%%%%%%%%%%%%%%%%%%%%%%%%%%%
\section{Discussion\label{sec:discussion}}

We have attempted to see how well we can assess the detector calibration for \ac{aLIGO} and 
\ac{AdV} using astrophysical sources. To do this we require that a \ac{sGRB}, for which it has 
been possible to measure the distance, is observed in coincidence with a \ac{CBC} signal in the 
\ac{GW} detectors. This enables us to assume a known distance to the source, and also limit the 
inclination of the source, which in turn allows us to test the consistency of the detector 
calibration. We do this by including an unknown scale factor on the true signal and noise, which we 
estimate given the data. We find that for detectable \ac{BNS} sources the uncertainty of on the 
calibration scale factor could on average be determined to $\lesssim 10\%$ of its true value for the 
\ac{aLIGO} and \ac{AdV} detectors out to 100\,Mpc. This is comparable to the hoped for level of 
calibration accuracy of the detectors. For sources at the standard single detector \ac{BNS} horizon 
distance of $\sim 450$\,Mpc the scale factor could on average be determined to within $\lesssim 
20\%$ of its true value for the \ac{aLIGO} detectors. Similar results were found for \ac{NSBH} 
sources, although sources could be observed out to higher distances.

The requirement of a coincident \ac{sGRB} with a known distance means that there will be 
considerable latency for this method of calibration assessment, but importantly this method would 
provide semi-independent consistency check that the calibration is reasonably accurate. We do note 
that this method has to make its own assumptions, including that the waveforms we use accurately 
reflect the signal, that the phase calibration does not change to much over frequency and that 
the overall calibration does not change considerably over the length of a signal. These are studies 
that could be incorporated in the future.

We imagine that a test for consistency of calibration could be performed with an ensemble of 
\ac{CBC} sources even without \ac{sGRB} counterparts, although such a method would have to 
marginalise over highly uncertain population models and may give quite weak bounds on the overall 
calibration uncertainty.

We emphasise that our work has assumed that the \ac{PSD} used in the likelihood function when 
estimating the calibration scale factor is an accurate representation of the \ac{PSD} at the time 
of the observed \ac{CBC} signal. In reality the \ac{PSD} may be estimated from a period close to, 
but not overlapping, the signal time and therefore may be slightly different 
\cite{2013PhRvD..88h4044L}. Our result would therefore be highly correlated with any 
uncertainty in the \ac{PSD} estimate, but would still offer an upper limit on the calibration 
uncertainty. However, future plans for \ac{CBC} parameter estimation include estimating the 
\ac{PSD} with the signal using BayesLine \cite{2015PhRvD..91h4034L}.

A further extension of this work would be for the calibration scale factor to be estimated as a 
function of frequency. This would require the frequency series to be broken up into sections and 
the calibration scale factor estimated for each. However, the scale factor uncertainty will 
obviously be related to the \ac{SNR} of a signal within a particular section. It could 
therefore be a job of Bayesian model selection to work out the optimal number of frequency 
sections to estimate the scale factor for. The BayesLine algorithm \cite{2015PhRvD..91h4034L}, or 
the method described in \cite{2013PhRvD..88h4044L}, may provide a natural way to perform such an
analysis. Indeed these methods, and those of \cite{Vitale:2012} are already being used for 
marginalising over calibration uncertainties, but do not attempt to estimate it.

\acknowledgements

We would like to acknowledge the useful discussions with a whole bunch of people. LW was part 
funded for this work through a Royal Astronomical Society Undergraduate Research Bursary.
MP is funded by the STFC under grant number ST/L000946/1.

\bibliography{masterbib}
\end{document}


