%%%%%%%%%%%%%%%%%%%%%%%%%%%%%%%%%%%%%%%%%%%%%%%%%%%%%%%%%%%%%%%%%%%%%%%%
%    INSTITUTE OF PHYSICS PUBLISHING                                   %
%                                                                      %
%   `Preparing an article for publication in an Institute of Physics   %
%    Publishing journal using LaTeX'                                   %
%                                                                      %
%    LaTeX source code `ioplau2e.tex' used to generate `author         %
%    guidelines', the documentation explaining and demonstrating use   %
%    of the Institute of Physics Publishing LaTeX preprint files       %
%    `iopart.cls, iopart12.clo and iopart10.clo'.                      %
%                                                                      %
%    `ioplau2e.tex' itself uses LaTeX with `iopart.cls'                %
%                                                                      %
%%%%%%%%%%%%%%%%%%%%%%%%%%%%%%%%%%
%
%
% First we have a character check
%
% ! exclamation mark    " double quote  
% # hash                ` opening quote (grave)
% & ampersand           ' closing quote (acute)
% $ dollar              % percent       
% ( open parenthesis    ) close paren.  
% - hyphen              = equals sign
% | vertical bar        ~ tilde         
% @ at sign             _ underscore
% { open curly brace    } close curly   
% [ open square         ] close square bracket
% + plus sign           ; semi-colon    
% * asterisk            : colon
% < open angle bracket  > close angle   
% , comma               . full stop
% ? question mark       / forward slash 
% \ backslash           ^ circumflex
%
% ABCDEFGHIJKLMNOPQRSTUVWXYZ 
% abcdefghijklmnopqrstuvwxyz 
% 1234567890
%
%%%%%%%%%%%%%%%%%%%%%%%%%%%%%%%%%%%%%%%%%%%%%%%%%%%%%%%%%%%%%%%%%%%
%
\documentclass[12pt]{iopart}
\usepackage {latexsym}
\usepackage {graphicx}
\usepackage{iopams}
\usepackage {color}

%\documentclass[12pt]{iopart}
%\newcommand{\gguide}{{\it Preparing graphics for IOP journals}}
%Uncomment next line if AMS fonts required
%\usepackage{iopams}  
%\usepackage {amsmath}
%\usepackage {amssymb}
%\usepackage {amsfonts}
%\usepackage {amsthm}
%\usepackage {mathrsfs}
%\usepackage {natbib}
%\usepackage {latexsym}
%\usepackage {graphicx}
%\usepackage {dsfont}
%\usepackage {times}
%\usepackage {txfonts}
%\usepackage {rotating}
%\usepackage {wasysym}
%\usepackage {multirow}
%\usepackage {hhline}
\usepackage {hyperref}
%\usepackage {color}
\usepackage {bm}
%\usepackage{appendix}
\usepackage{acronym}
%\usepackage{dcolumn}   % needed for some tables
\usepackage {url}
%\usepackage[normalem]{ulem}   % temporal one in draft.

% variable shortcuts 
\newcommand{\Hub}{H_{0}}
\newcommand{\DL}{D_{\mathrm{L}}}
\newcommand{\lam}{\bm{\lambda}}
%\newcommand{\data}{\bm{d}}
%\newcommand{\sigpar}{\bm{\theta}}
%\newcommand{\nuispar}{\bm{\lambda}}
%\newcommand{\globpar}{\bm{\gamma}}
%\newcommand{\model}{\mathcal{H}}

%% ----- input git-version tag
\input{tag.tex}

\newcommand{\dcc}{LIGO-PXXXXXXXX}
\newcommand{\CM}[1]{\textcolor{red}{CM: #1}}
\newcommand{\MH}[1]{\textcolor{blue}{MH: #1}}
\newcommand{\XF}[1]{\textcolor{green}{XF: #1}}

\newcommand{\aap}{A\&A }
\newcommand{\apj}{ApJ  }
\newcommand{\apjl}{ApJL  }
\newcommand{\apjs}{ApJS  }
\newcommand{\prd}{Phys. Rev. D  }
\newcommand{\nat}{Nature}
\newcommand{\araa}{ARA\&A}
\newcommand{\mnras}{Monthly Notices of the Royal Astronomical Society  }
\begin{document}

\title{Requirements and usefulness of galaxy catalogues for gravitational wave astrophysics}

\author{L.~Wright$^1$, M.~Pitkin$^1$ \& C.~Messenger$^1$}
\address{$^1$ SUPA, School of Physics and Astronomy, University of
  Glasgow, Glasgow G12 8QQ, United Kingdom}
\ead{1002990W@student.gla.ac.uk}

\begin{abstract}
  We present an investigation into the potential of calibrating
  gravitational wave detector outputs through the use of standard
  sirens. Such signals, as measured via gravitational wave
  observations provide an estimated luminosity distance which is
  subject to uncertainties in the calibration of the data.  If a host
  galaxy is identified for a given source then its redshift can be
  obtained and combined with current knowledge of the cosmologic
  parameters.  This will yield the true luminosity distance and allow
  the direct comparison with the estimated value.  Discrepancies can
  then be used to correct the originbal calibration.
\end{abstract}

% acronym definitions
\acrodef{GW}[GW]{gravitational wave}
\acrodef{BNS}[BNS]{binary neutron star}
\acrodef{MWEG}[MWEG]{Milky Way equivelent galaxies}
\acrodef{BBH}[BBH]{binary black hole}
\acrodef{NSBH}[NSBH]{neutron star black hole}
\acrodef{SNR}[SNR]{signal-to-noise ratio}
\acrodef{GRB}[GRB]{gamma-ray burst}

%Uncomment for PACS numbers title message
%\pacs{00.00, 20.00, 42.10}
% Keywords required only for MST, PB, PMB, PM, JOA, JOB? 
%\vspace{2pc}
%\noindent{\it Keywords}: Article preparation, IOP journals
% Uncomment for Submitted to journal title message
%\submitto{\JPA}
% Comment out if separate title page not required
\maketitle

%%%%%%%%%%%%%%%%%%%%%%%%%%%%%%%%%%%%%%%%%%%%%%%%%%
%%%%%%%%%%%%%%%%%%%%%%%%%%%%%%%%%%%%%%%%%%%%%%%%%%
\section{Introduction\label{sec:intro}}

Explain the basic idea.

%%%%%%%%%%%%%%%%%%%%%%%%%%%%%%%%%%%%%%%%%%%%%%%%%%
%%%%%%%%%%%%%%%%%%%%%%%%%%%%%%%%%%%%%%%%%%%%%%%%%%
\section{Gravitational wave detector calibration\label{sec:calibration}}

We need to define our model for the calibration, i.e. are we talking
about amplitude only and are we breaking this up over frequency.

%%%%%%%%%%%%%%%%%%%%%%%%%%%%%%%%%%%%%%%%%%%%%%%%%%
%%%%%%%%%%%%%%%%%%%%%%%%%%%%%%%%%%%%%%%%%%%%%%%%%%
\section{Binary neutron star standard sirens\label{sec:sirens}}

We should explain what this means starting with reference
to~\cite{1986Natur.323..310S}.

%%%%%%%%%%%%%%%%%%%%%%%%%%%%%%%%%%%%%%%%%%%%%%%%%%
%%%%%%%%%%%%%%%%%%%%%%%%%%%%%%%%%%%%%%%%%%%%%%%%%%
\section{GRB counterparts\label{sec:GRB}}

We should discuss the key points with regards to 

%%%%%%%%%%%%%%%%%%%%%%%%%%%%%%%%%%%%%%%%%%%%%%%%%%
%%%%%%%%%%%%%%%%%%%%%%%%%%%%%%%%%%%%%%%%%%%%%%%%%%
\section{A single event with GRB counterpart\label{sec:single}}

This should be the simplest case we can do (and maybe the only case).

%%%%%%%%%%%%%%%%%%%%%%%%%%%%%%%%%%%%%%%%%%%%%%%%%%
%%%%%%%%%%%%%%%%%%%%%%%%%%%%%%%%%%%%%%%%%%%%%%%%%%
\section{Multiple events without counterpart\label{sec:multiple}}

An analysis or discussion of the issues regarding multiple events
without EM counterparts.

%%%%%%%%%%%%%%%%%%%%%%%%%%%%%%%%%%%%%%%%%%%%%%%%%%
%%%%%%%%%%%%%%%%%%%%%%%%%%%%%%%%%%%%%%%%%%%%%%%%%%
\section{Discussion\label{sec:discussion}}

What did we learn.

\ack

We would like to acknowledge the useful discussions with a whole bunch
of people.

\section*{References}

\bibliographystyle{unsrt}
\bibliography{masterbib}
\end{document}


