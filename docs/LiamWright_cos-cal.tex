\documentclass[a4paper]{article}

\title{Modifications to Calibration files}
\author{Liam Wright}

\begin{document}
\maketitle

\section*{Initial}

Modifications to the inspiraltest.m file:
\begin{enumerate}
   \item Move definition of $l2p = -\frac{1}{2}log(2\pi)$ from draw\_mcmc\_sample.m to mcmc\_sampler.m
   \item Changing scale frequencies from a range of $F_{min}$ to $F_{max}$ to 125Hz.
   \item In freqdomaininspiral.m had to change the scaleint to work for only one scale frequency.
   \item Running the code with every variable as a set value, and then continuely adding more variables into the prior; each variable being:
   \begin{enumerate}
      \item Masses of each spiraling star having a value between 1 and 2.5 $M_{\odot}$. Uniform prior.
      \item psi having a uniform prior with a value from $-\frac{\pi}{4}$ to $\frac{\pi}{4}$
      \item phic having a uniform prior with a value from 0 to $2\pi$
      \item Luminosity distance having a gaussian prior with a width of about 10\% of its injection distance.
      \item Iota value with a gaussian prior. Width about 30\% of a radian and an injection angle of 0 radians.

   \end{enumerate}
   Remebering to change update in extraparams(...) to 1. \\

   Comparing the standard deviations of post\_samples(:,1), the scale frequency.

\end{enumerate}


Next I used the effective\_sample\_size.m to work out the autocorrected length and effective number of uncorrelated samples in the post\_samples matrix with N parameters (same  from the prior). \\





\end{document}
